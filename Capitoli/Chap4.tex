\chapter{Conclusioni}
\label{cap:Conclusioni}
L'obiettivo del lavoro è stato quello di ottenere una pipeline funzionante che potesse mettere a disposizione, incorporandole nelle mappe, informazioni aggiuntive riguardanti le facciate degli edifici attraverso le point cloud che le descrivono.\\
Il sistema è risultato funzionante e facilmente riproducibile, il lavoro potrà quindi essere utilizzato per testare se esiste un miglioramento sostanziale dell'accuratezza di localizzazione attraverso l'utilizzo di mappe aumentate. Durante la prototipazione della pipeline sono stati fatti alcuni passaggi manuali che possono essere facilmente automatizzati con l'ausilio di script dedicati che potranno essere svilupppati in seguito, in particolare l'aggiunta del tag che rappresenta le facciate all'interno di OSM è facilmente automatizzabile attraverso l'utilizzo di librerie dedicate.\\
La parte di segmentazione ha dato risultati ottimali solo su specifici dataset con poche fonti di disturbo come alberi e segnali stradali, per tanto è necessario trovare e utilizzare un modello per la segmentazione delle facciate più robusto, in grado di funzionare in una maggior quantità di scene. 
Inoltre si potrebbero sottomettere le modifiche effettuate alle mappe "globali" in modo da eliminare la necessità di utilizzarne una loro istanza locale, ciò ha tuttavia problemi legislativi ed è, al momento, impossibile.
